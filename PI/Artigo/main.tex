%%%%%%%%%%%%%%%%%%%%%%%%%%%%%%%%%%%%%%%%%%%%%%%%%%%%%%%%%%%%%%%%%%%%%%
% How to use writeLaTeX: 
%
% You edit the source code here on the left, and the preview on the
% right shows you the result within a few seconds.
%
% Bookmark this page and share the URL with your co-authors. They can
% edit at the same time!
%
% You can upload figures, bibliographies, custom classes and
% styles using the files menu.
%
%%%%%%%%%%%%%%%%%%%%%%%%%%%%%%%%%%%%%%%%%%%%%%%%%%%%%%%%%%%%%%%%%%%%%%

\documentclass[12pt]{article}

\usepackage{sbc-template}

\usepackage{graphicx,url}

%\usepackage[brazil]{babel}   
\usepackage[utf8]{inputenc}  

     
\sloppy

\title{IoT de Baixo Custo com ESP32 e MQ-3 para Redução de Perdas de FLV em recipientes dedicados}

\author{Pablo Henrique L. de Oliveira, Luca Atanazio Evangelista}


\address{Faculdade Senai Fatesg
  (Fatesg)\\
  Rua 227-A, Qd. 117, Lt. Área n° 582, Setor Leste Universitário\\
CEP 74610-060 - Goiânia-GO
\nextinstitute
  Curso Superior de Tecnologia em Inteligência Artificial - SENAI Fatesg, GO
  \email{eopabrogmail.com, lucaatanazioevangelista@gmail.com}
}
\begin{document} 

\maketitle

\begin{abstract}
The high loss rate of Fruits, Legumes, and Vegetables (FLV) in Brazilian retail, which reached 5.83\% in 2024, represents a major challenge due to operational inefficiency. Climatic products such as tomatoes and bananas are the main culprits, contributing to accelerated losses caused by the release of ethylene under ambient and storage conditions. This work proposes the development of a low-cost Internet of Things (IoT) monitoring system using sensors (temperature, humidity, and volatile gases) installed on shelves. The system's objective is to transform raw data into valuable information to support preventive measures, enabling rigorous control of the display environment. This technology is expected to reduce losses, increase shelf life, and improve operational efficiency and sustainable management of FLV.\\

Keywords: Internet of Things; Postharvest Losses; Climacteric Fruits; Retail Management; Ethylene.

\end{abstract}
     
\begin{resumo} 
 A alta taxa de perdas em Frutas, Legumes e Verduras (FLV) no varejo brasileiro  representa um grande desafio de ineficiência operacional. Produtos climatéricos como o tomate e a banana são os principais responsáveis, contribuindo com perdas aceleradas causado pela liberação de etileno sob condições de ambiente e de armazenamento. Este trabalho propõe o desenvolvimento de um Sistema de Monitoramento de baixo custo em Internet das Coisas (IoT), utilizando sensores (temperatura, umidade e gases voláteis) instalados nas gôndolas. O objetivo do sistema é transformar dados brutos em informações importantes para auxílio em medidas preventivas, permitindo o controle rigoroso do ambiente de exposição. Espera-se que esta tecnologia reduza as perdas, aumente a vida de prateleira e melhore a eficiência operacional e a gestão sustentável de FLV.\\

Palavras-chave: Internet das Coisas; Perdas Hortifruti; Frutos Climatéricos; Gestão de Varejo; Etileno.
\end{resumo}


\section{Introdução}

O Brasil, é uma potência agrícola global, que apresenta uma produção expressiva.Segundo o IBGE, Em 2024, a safra de frutas como a maçã atingiu 997.470 toneladas e a de banana alcançou 7.046.502 toneladas em âmbito nacional. Regionalmente, o estado de Goiás é destaque na produção de tomate, com um volume superior a 4 milhões de toneladas(IBGE, 2025). No entanto, essa produção enfrenta desafios de perdas ao longo do processo, especialmente no varejo. Segundo a Associação Brasileira de Supermercados (ABRAS, 2025), o setor de Frutas, Legumes e Verduras (FLV) registrou uma ineficiência operacional de 5,83\% em 2024, um índice superior à média geral do supermercado, que oscilou entre 1,79\% e 1,89\% no período de 2018 a 2023.

A análise da ABRAS revela que a ineficiência no setor de FLV não é um problema recente, com registros de 5,01\% em 2019, demonstrando frutas como as principais responsáveis por essas perdas, por 27\% do total. Dentro do grupo de frutas, legumes e verduras, o tomate e a banana surgem como os produtos mais riscos, líderes de perdas tanto em volume (quilos) quanto em valor. A causa fundamental para essa problema no setor de hortifrúti consiste em suas características fisiológicas, sendo um deles o efeito climatérico. Frutos climatéricos, como maçã, banana e tomate, liberam etileno ($C_2H_4$), um hormônio vegetal gasoso que acelera o processo de amadurecimento, não apenas em si mesmos, mas também em outros produtos sensíveis ao gás que estejam armazenados ou expostos no mesmo ambiente.

Este processo bioquímico é intensificado por condições ambientais inadequadas, principalmente por temperatura. Cada produto possui uma condição de temperatura ideal para sua conservação, que podem acelerar as taxas respiratórias e a produção de etileno, causando a degradação da qualidade e reduzindo a o tempo de prateleira de todo o setor. A falta de um monitoramento em tempo real desses dados no ponto de venda dos supermercados impede a tomada de ações proativas, resultando em perdas significativas.

Diante deste cenário, este trabalho propõe o desenvolvimento e a aplicação de um sistema de monitoramento de baixo custo baseado em tecnologia de Internet das Coisas (IoT). O sistema utiliza sensores para coletar dados brutos e contínuos de temperatura, umidade e gases associados à maturação e deterioração — como o etileno, detectado por sensores de gases diretamente das gôndolas de supermercados. O objetivo é transformar esses dados em informações, permitindo um controle do ambiente de exposição e a implementação de soluções estratégicas para reduzir as perdas, com foco especial nos produtos de alto risco como tomate e banana e facilidade de detecção por equipamentos de baixo custo. Este artigo detalha a arquitetura do sistema, a metodologia de coleta e análise de dados e discute como essa tecnologia pode impactar a gestão de FLV no varejo, aumentando a eficiência operacional e a sustentabilidade.

\section{Fundamentação teórica}
O setor de Frutas, Legumes e Verduras (FLV) no varejo de supermercados apresenta uma alta nos índices de ineficiência operacional. Dados da Associação Brasileira de Supermercados (ABRAS, 2025) indicam que a ineficiência operacional média do setor atingiu 5,83\% no último ciclo de pesquisa. Este valor é superior à média geral de ineficiência do supermercado, que, apesar de oscilações, consolidou-se em 1,89\% nos períodos mais recentes, e já foi historicamente referenciada em torno de 1,78\% em edições anteriores. Tal contraste evidencia uma necessidade de intervenção tecnológica e estratégica para mitigar as perdas no setor de FLV, cujo índice de ineficiência é mais do que o triplo da média, reforçando a urgência por um monitoramento.

A problemática da ineficiência operacional no setor de Frutas, Legumes e Verduras (FLV) evidencia a aceleração da degradação dos alimentos, diretamente ligada ao efeito climatérico e à emissão do hormônio gasoso Etileno ($\text{C}_2\text{H}_4$). Frutos como a banana, maçã e tomate são classificados como climatéricos e definem-se pelo aumento no pico respiratório durante o período de maturação. Esse pico coincide com a produção endógena de Etileno. O $\text{C}_2\text{H}_4$ atua como um hormônio vegetal que se amplifica; ou seja, a produção de uma pequena quantidade do gás estimula a emissão de uma quantidade significativamente maior pelo próprio fruto. Este processo acelera a degradação da clorofila, o amolecimento da polpa e a transformação de amido em açúcares. A principal consequência no ambiente de varejo é que a liberação de $\text{C}_2\text{H}_4$ por um fruto maduro atua sobre os frutos próximos, catalisando o amadurecimento em cadeia e levando a perdas rápidas na gôndola. A velocidade de ação e produção do efeito catalítico do etileno são reguladas por fatores ambientais, como a temperatura e a umidade relativa: Temperatura, acima da faixa ideal de conservação, a taxa respiratória do fruto aumenta, elevando a produção de $\text{C}_2\text{H}_4$. Este aumento térmico precipita o ciclo de vida do fruto, reduzindo sua vida útil de prateleira; Umidade Relativa, a baixa umidade provoca a perda de água e o murchamento, comprometendo a textura do alimento. Em contrapartida, a alta umidade, quando combinada a temperaturas elevadas, favorece o desenvolvimento de fungos e patógenos, resultando em deterioração e descarte.

Desse modo, a adoção de tecnologias de baixo custo e a arquitetura Edge Computing são fundamentais para a acessibilade economica e a escalabilidade de monitoramento em ambientes de varejo. Os sensores DHT11 (Temperatura/Umidade) e MQ-3 (Gases Voláteis) são selecionados com base nos seguintes critérios: Viabilidade Econômica, onde sensores dedicados para a detecção de etileno possuem custo elevado para implantação em grande escala. O uso do MQ-3 para Etanol e outros Componentes Orgânicos Voláteis (VOCs) associados à deterioração e ao amadurecimento permite um custo reduzido por ponto de monitoramento, apresentando a melhor alternativa de baixo custo; Acessibilidade e suporte, no qual a disponibilidade da série MQ e do DHT11 no mercado facilita a manutenção, a substituição e a prototipagem rápida; Suficiência de Dados, onde para o objetivo do projeto (classificação de status da fruta), o fornecimento de dados de temperatura/umidade (DHT11) e do valor analógico  suficiente para bruto do gás (gasValor do MQ-3) éalimentar um algoritmo de regras ou um modelo preditivo. Além disso, o sistema adota o modelo de Edge Computing, onde o microcontrolador ESP32 atua como o Módulo Edge. Esta arquitetura é essencial para a eficiência operacional por: Redução da latência, pois seu envio direto de dados via HTTP POST garante que os alertas de status da fruta sejam gerados e transmitidos rapidamente ao backend. A inteligência na borda minimiza a latência ao formatar o payload JSON antes da transmissão; Eficiência de rede, onde o sistema envia apenas o payload JSON estruturado em intervalos definidos (conforme o delay no código), o qual otimiza o uso da banda Wi-Fi e reduz a carga de trabalho no servidor central; Autonomia e resiliência, pois o módulo Edge pode ser programado (em trabalhos futuros) para armazenar dados localmente em caso de falha de conectividade, garantindo a resiliência do sistema e a não interrupção da coleta de dados. Portanto, a combinação estratégica do hardware de baixo custo com a inteligência distribuída da arquitetura Edge resulta em uma solução que é simultaneamente economicamente viável, robusta e escalável para o monitoramento contínuo da cadeia de FLV.

A respeito dos trabalhos relacionados do projeto, a pesquisa sobre sistemas de monitoramento de qualidade e amadurecimento de frutos climatéricos via internet das coisas é uma área crescente para abordar o tema para as redes de comércio, como o objetivo desta aplicação em hortífrutis. Dentre os principais trabalhos acadêmicos que utilizam sensores de baixo custo para a detecção de gases voláteis associadas à degradação de Frutas, Legumes e Verduras (FLV) estão:

*\textbf{ Sistema de IoT para Monitoramento da Maturação de Frutas por Cor e Gás Etileno} - Uso do \textbf{MQ-3 (e Sensor de Cor)} - Frutos em estudo \textbf{Variado} - Algoritmo/Modelo voltado a \textbf{Rede Neural (IA)} - Foco na \textbf{Classificação de maturação via cor e gás.
Altamente relavante por também empregar o ESP32 e o sensor MQ-3 para monitoramento}. Contudo, seu principal foco reside na aplicação de uma Rede neural em um modelo que integra cor e gás, tornando o sistema potencialmente mais complexo de ser implementado e calibrado no ambiente dinâmico do varejo.

* \textbf{Response Characteristics Study of Ethylene Sensor for Fruit Ripening under Temperature Control} - Uso do \textbf{Sensor dedicado/calibrado} - Frutos em estudo \textbf{Abacate} - Algoritmo/Modelo voltado a \textbf{Análise de Resposta} - Foco na \textbf{Validação da dependência do sensor de etileno com a temperatura}.
Fornece a base cientifica crucial para o nosso projeto, ao validade a forte correlação entre a resposta do sensor de etileno e as variações de temperatura. Este estudo justifica a escolha do nosso sistema em integrar a leitura do DHT11 e do MQ-3 para uma avaliação se status mais precisa.

* \textbf{IOT SENSOR-BASED GAS DETECTION DEVICE FOR STORED FRUITS} - Uso do \textbf{MQ-4/MQ-2 (VOCs)} - Frutos em estudo \textbf{Banana, Mamão, Manga} - Algoritmo/Modelo voltado a \textbf{Regras Simples - Foco na Detecção de deterioração e amadurecimento em armazenamento}.
* \textbf{smart banana quality tracking and monitoring system via wifi with database} - Uso do \textbf{MQ-3} - Frutos em estudo \textbf{Banana} - Algoritmo/Modelo \textbf{Não Detalhado} - Foco na \textbf{Rastreamento da qualidade em tempo real}. 
Os dois últimos trabalhos demonstram a eficácia da abordagem IoT com sensores da série MQ e foca em frutos climatéricos (banana e manga). No entanto, o principal diferencial do nosso sistema reside em: O foco operacional e econômico, justificado e focado na redução do índice de 5,83\% de perdas no varejo(conforme os dados da pesquisa da \textbf{ABRAS}), oferecendo uma solução de monitoramento na gôndola do ponto de venda, e não primariamente em câmeras de armazenamento; Arquitetura otimizada, foi-se adotado a arquitetura de Edge Computing, priorizando a baixa latência e eficácia de rede, visto ser essencial para gerar alertas de "Venda Rápida" em tempo hábil para a equipe do supermercado. Utilizamos um algoritimo de regras, visando a facilidade de implantação e manutenção em escala no setor varejista.

\section{Metodologia}
\label{sec:metodologia}

O desenvolvimento do sistema de monitoramento de hortifrúti segue uma metodologia de prototipagem ágil, focando na integração de hardware de baixo custo e um algoritmo de análise de dados baseado em regras. Esta seção detalha a arquitetura do sistema, os componentes físicos utilizados e a lógica de processamento de dados embarcada e no servidor. 
O sistema proposto adota uma arquitetura de Internet das Coisas (IoT) em camadas, seguindo o modelo \textbf{Edge Computing}, o qual é projetado para otimizar a coleta de dados e a tomada de decisão em tempo real. A arquitetura é composta por três camadas principais que ditam o fluxo da informação: Módulo Edge, Comunicação e Processamento, e Frontend/Alerta.

\subsubsection{Módulo Edge (Dispositivo de Coleta)}

O \textbf{Módulo Edge} é o ponto de coleta instalado na gôndola do varejo, sendo responsável por ler e pré-processar os dados brutos.

\begin{itemize}
    \item \textbf{Componente Central:} Utiliza o microcontrolador \textbf{ESP32}, escolhido pela sua capacidade de processamento e conectividade Wi-Fi embarcada.
    \item \textbf{Sensores:} O módulo integra o sensor de Temperatura e Umidade \textbf{DHT11} e o sensor de Gases Voláteis \textbf{MQ-3}.
    \item \textbf{Função:} O ESP32 realiza a leitura contínua dos dados (\textit{temperatura}, \textit{umidade} e \textit{gasValor}) e os empacota em um \textit{payload} estruturado (formato JSON), preparando-os para o envio.
    \item \textbf{Conexão de Hardware:} A montagem física seguiu um esquema de conexão direta: o pino de dados do DHT11 foi conectado à porta GPIO 4 do ESP32, e a saída analógica do MQ-3 (para leitura do $\text{gasValor}$ bruto) foi conectada à porta GPIO 34.
    \item \textbf{Dependências de Software:} O software embarcado requer a utilização da biblioteca ´DHT.h´ e sua dependência, ´Adafruit Unified Sensor´, para a correta interface com o sensor ambiental.
   
\end{itemize}

\subsubsection{Comunicação e Processamento (\textit{Backend})}

Esta camada é responsável pela transmissão segura dos dados e pela aplicação da lógica de negócio.

\begin{itemize}
    \item \textbf{Comunicação:} Os dados são enviados do Módulo Edge para o servidor central através de uma requisição \textbf{HTTP POST}, em frequência detalhada na seção \textbf{3.0.4}.
    \item \textbf{Servidor e Processamento:} O servidor (\textit{Backend}) recebe o \textit{payload} JSON, normaliza os dados brutos e executa o \textbf{Algoritmo de Decisão/IA}.
    \item \textbf{Análise:} É neste ponto que o sistema aplica as regras de classificação (baseadas nos \textit{thresholds} de $\text{C}_2\text{H}_4$ \textit{proxy}, temperatura e umidade) para determinar o \textit{status} da fruta (ex: "Verde", "Madura - Venda Rápida", "Descarte").
\end{itemize}

\subsubsection{Frontend e Alerta (Consumidor de Dados)}

O resultado da análise do \textit{Backend} é transformado em informação acionável, permitindo a gestão proativa das perdas.

\begin{itemize}
    \item \textbf{Visualização:} Os \textit{status} classificados são exibidos em um painel (\textit{Dashboard}), permitindo a visualização em tempo real das condições de conservação.
    \item \textbf{Tomada de Decisão:} O sistema gera alertas quando o \textit{status} atinge o \textit{threshold} de "Venda Rápida" ou "Descarte", permitindo que a equipe tome medidas operacionais proativas (realocação ou mudança de preço).
\end{itemize}

\subsubsection{Comunicação e Frequência de Coleta)}

A comunicação com o servidor \textit{backend} é realizada através de requisições \textsc{HTTP POST} seguras via Wi-Fi. O \textsc{envio do \textit{payload} JSON} completo para o servidor é controlado por um temporizador ´millis()´ e ocorre em \textsc{intervalos fixos de 10 segundos}, embora o \textit{loop} principal do código do ESP32 realize a leitura dos sensores a cada 2 segundos para o pré-processamento interno. Esta frequência otimiza o uso da energia e tráfego da rede, mantendo o sistema responsivo

\section{Resultados e Discussão}
\label{sec:resultados}



\begin{document}

\section{Referências}

\noindent
INSTITUTO BRASILEIRO DE GEOGRAFIA E ESTATÍSTICA (IBGE). \textit{Levantamento Sistemático da Produção Agrícola (LSPA). Estimativa de Janeiro 2025 para a Safra Nacional.} IBGE, 2025. Disponível em: \url{https://ftp.ibge.gov.br/Producao_Agricola/Levantamento_Sistematico_da_Producao_Agr...}. Acesso em: 10 nov. 2025.

\vspace{0.5cm}

\noindent
ASSOCIAÇÃO BRASILEIRA DE SUPERMERCADOS (ABRAS). \textit{Pesquisa de Eficiência Operacional: Resultados 2025.} São Paulo: ABRAS, 2025. Disponível em: \url{https://www.abras.com.br/economia-e-pesquisa/pesquisa-de-eficiencia-operacional/pesquisa-2025}. Acesso em: 10 nov. 2025.

\vspace{0.5cm}

\noindent
SILVA, Jorge. \textit{Sistema de IoT para Monitoramento da Maturação de Frutas por Cor e Gás Etileno.} Trabalho de Conclusão de Curso (Graduação em Engenharia de Computação) - Universidade Federal do Amazonas (UFAM), Manaus, 2020. Disponível em: \url{https://riu.ufam.edu.br/bitstream/prefix/7931/5/TCC_JorgeSilva.pdf}. Acesso em: 17 nov. 2025.

\vspace{0.5cm}

\noindent
WANG, H. et al. Response Characteristics Study of Ethylene Sensor for Fruit Ripening under Temperature Control. \textit{Sensors}, Basel, v. 23, n. 11, p. 5203, 2023. Disponível em: \url{https://www.mdpi.com/1424-8220/23/11/5203}. Acesso em: 17 nov. 2025.

\vspace{0.5cm}

\noindent
CHANDRASEKARAN, N. et al. IOT SENSOR-BASED GAS DETECTION DEVICE FOR STORED FRUITS. \textit{Plant Archives}, v. 22, n. 1, p. 750-756, 2022. Disponível em: \url{https://www.plantarchives.org/article/108-%20IoT%20Sensor-Based%20Gas%20Detection%20Device%20for%20Stored%20Fruits%20750-756%20(Sp-103).pdf}. Acesso em: 17 nov. 2025.

\vspace{0.5cm}

\noindent
ISMAIL, A. H. et al. Smart banana quality tracking and monitoring system via wifi with database. In: \textit{International Conference on Computer and Drone Application (IConDA)}. MARA, 2021. Disponível em: \usepackage{url}{https://www.mara.gov.my/wp-content/uploads/2025/09/44.-IPPM-SMART-BANANA-QUALITY-TRACKING-AND-MONITORING-SYSTEM-via-WIFI-WITH-DATABASE-MJII.pdf}. Acesso em: 17 nov. 2025.
\end{document}


\end{document}
